\documentclass{article}
\usepackage[utf8]{inputenc}
\usepackage[T1]{fontenc}
\usepackage[french]{babel}
\usepackage{graphicx}

% BIBLATEX
\usepackage[defernumbers=false, backend=biber, sorting=nyt, style=verbose,maxbibnames=99]{biblatex}

% remove language field from all entries
\AtEveryBibitem{%
  \clearlist{language}%
}

% replacing <In:> by <,> (TRUE = @article) or dans (FALSE)
% source : https://tex.stackexchange.com/questions/10682/suppress-in-biblatex
% \renewbibmacro{in:}{, dans\intitlepunct{}} % everywhere
% ifentrytype{entry}{<TRUE>}{<FALSE>}
\renewbibmacro{in:}{%
  \ifentrytype{article}{,\space}{\addcomma \space dans\intitlepunct}
  }
  
% replacing the coma between last <author> and <editortype> by a blank space
\renewcommand{\editortypedelim}{\space}

% adding coma as delimiter between <title> and <byeditor>.
% source : https://tex.stackexchange.com/questions/261275/how-do-you-change-the-editor-field-in-bibliography-using-biblatex
\renewbibmacro*{byeditor+others}{%
  \setunit{,\space}%
  \ifnameundef{editor}
    {}
    {\usebibmacro{byeditor+othersstrg}%
     \setunit{\addspace}%
     \printnames[byeditor]{editor}%
     \clearname{editor}%
     \newunit}%
  \usebibmacro{byeditorx}%
  \usebibmacro{bytranslator+others}%
  \printunit*{,\newunitpunct}}%

% adding a coma between  <editortype> and <title>
% source : https://tex.stackexchange.com/questions/552046/biblatex-what-is-the-delimiter-between-author-and-title-in-the-bibliography-for
\DeclareDelimFormat[bib]{nametitledelim}{,\space}

% customize the dash to be used as a replacement for recurrent authors or editors :
% \renewcommand{\bibnamedash}{}

\DeclareFieldFormat[book,proceedings]{title}{\textit{#1},}
\DeclareFieldFormat[*]{series}{Coll. \og #1 \fg,}
\DeclareFieldFormat[*]{volume}{vol. #1,}
\DeclareFieldFormat[*]{note}{#1,}

% replacing colon by coma between <location> and <publisher>
% source https://tex.stackexchange.com/questions/187456/changing-colon-to-comma-after-conference-address-with-biblatex
\renewbibmacro*{publisher+location+date}{%from standard.bbx
  \printlist{location}%
  \iflistundef{publisher}
    {\setunit*{\addcomma\space}}
    {\setunit*{\addcomma\space}}%instead of \addcolon\space
  \printlist{publisher}%
  \setunit*{\addcomma\space}%
  \usebibmacro{date}%
  \newunit}

% inversing editor and booktitle in @incollection
% source : https://tex.stackexchange.com/questions/528989/biblatex-rearrange-editor-and-book-title-in-incollection-entry
\DeclareNameAlias{sortname}{family-given}
\DeclareNameAlias{default}{family-given}
\DeclareFieldFormat{editortype}{\mkbibparens{#1}}
\DeclareDelimFormat{editortypedelim}{\addspace}
\renewcommand*{\labelnamepunct}{\addcomma\addspace}
\DeclareDelimFormat{innametitledelim}{\labelnamepunct}
\DeclareNameAlias{ineditor}{default}
\newbibmacro*{bbx:in:editor}[1]{%
  \ifboolexpr{
    test \ifuseeditor
    and
    not test {\ifnameundef{editor}}
  }
    {\printnames[ineditor]{editor}%
     \setunit{\printdelim{editortypedelim}}%
     \usebibmacro{#1}%
     \clearname{editor}}
    {}}
\newbibmacro*{in:editor}{%
  \usebibmacro{bbx:in:editor}{editorstrg}}
\newbibmacro*{in:editor+others}{%
  \usebibmacro{bbx:in:editor}{editor+othersstrg}}
\usepackage{xpatch}
\xpatchbibdriver{incollection}
  {\usebibmacro{maintitle+booktitle}%
   \newunit\newblock}
  {\usebibmacro{in:editor+others}%
   \setunit{\printdelim{innametitledelim}}\newblock
   \usebibmacro{maintitle+booktitle}%
   \printunit{\addcomma\space}\newblock}
  {}{}

% CSQUOTES
\usepackage[autostyle]{csquotes}

% URLs
\usepackage[colorlinks=true,linkcolor=black,anchorcolor=black,citecolor=black,filecolor=black,menucolor=black,runcolor=black,urlcolor=blue]{hyperref}
\usepackage{url}

% BIB SOURCE
\addbibresource{biblio2017.bib}

% Cut URLs in bibliography
\setcounter{biburllcpenalty}{7000}
\setcounter{biburlucpenalty}{8000}

% HEADER & FOOTER
% source : https://tex.stackexchange.com/questions/252411/subsection-name-as-right-header/252415
\usepackage{fancyhdr}
\pagestyle{fancy}
\renewcommand{\headrulewidth}{0pt} % enlever la règle
\renewcommand{\sectionmark}[1]{\markboth{#1}{}}
\fancyhf{}
\fancyhead[R]{\small{\textit{\thesection . \leftmark}}}
\cfoot{\thepage}

% BLANK PAGE
\usepackage{afterpage}
\newcommand\blankpage{%
    \null
    \thispagestyle{empty}%
    \addtocounter{page}{-1}%
    \newpage}

% IMAGES (EHESS AND CNRS TITLE PAGE)
\usepackage{mwe}

% LOGO BIBTEX
\usepackage{hologo}

\title{\textbf{Bibliographie annuelle}}
\author{Centre de recherches historiques\footnote{\textit{Document établi en mars 2021 par Jean-Damien Généro, ingénieur d'études du CNRS, dans le cadre du collectif Sources et données de la recherche du Centre de recherches historiques ; reprise du fichier initial constitué par Cécile Soudan, ingénieure de recherche du CNRS, sous la supervision de Béatrice Delaurenti, maître de conférence de l'EHESS, et de 
Thomas Le Roux, chargé de recherche du CNRS, directrice et directeur du CRH.}}}
\date{1\up{er} janvier -- 31 décembre 2017}

\begin{document}
\renewcommand{\contentsname}{Sommaire}
%\maketitle
\begin{titlepage}
\begin{center}

\vspace*{1,75cm}

\includegraphics[width=4cm]{img/logo_crh_magenta.png}
\bigskip
\bigskip
\bigskip
\bigskip

\begin{Huge}
\textbf{Bibliographie annuelle}
\end{Huge}

\bigskip

\begin{LARGE}
\textbf{1\up{er} janvier -- 31 décembre 2019}

\bigskip
\bigskip


\textbf{\emph{Centre de recherches historiques} }\\
\end{LARGE}

\bigskip
\bigskip


\vspace*{3cm}

\includegraphics[width=2cm]{img/licence-cc-icon.png}

\vspace*{3,5cm}

\begin{tabular}{cc}
    \includegraphics[width=2cm]{img/Logo_EHESS_2021_RVB.png} & \includegraphics[width=2cm]{img/cnrslogo.png} \\
\end{tabular}

\end{center}
\afterpage{\blankpage}
\end{titlepage}


\newpage
\thispagestyle{empty}

\begin{figure}
    \centering
    \includegraphics{img/licence-cc-icon.png}
    \label{fig:licence}
\end{figure}

\bigskip

\bigskip
\begin{center}
\begin{itshape}

Document compilé en avril 2021 avec le langage \LaTeX. Les fichiers \hologo{BibTeX} contenant l'ensemble des données sont également disponibles.

\medskip

Sous la supervision de la direction du CRH : Béatrice Delaurenti (maître de conférence de l'EHESS) et Thomas Le Roux (chargé de recherche du CNRS).

\medskip

Coordination par Cécile Soudan (ingénieure de recherche du CNRS) et Jean-Damien Généro (ingénieur d'études du CNRS).

\medskip

Conception technique et gestion des données : Jean-Damien Généro.

\medskip

Saisie et suivi du projet : Cécile Soudan, Jean-Damien Généro, Francine Filoche (technicienne de l'EHESS) et Caroline Long Baros (assistante ingénieure de l'EHESS).

\medskip

Pour toute demande, contacter :\\\url{sources_et_donnees.crh@ehess.fr}.

\end{itshape}

\bigskip

\bigskip

    \textbf{Document placé sous les termes de la licence Creative Commons Attribution - Pas d'Utilisation Commerciale - Pas de Modification 4.0 International (CC BY-NC-ND 4.0).}
\end{center}

\bigskip

    Vous êtes autorisés à :
    \begin{itemize}
        \item \textbf{Partager} --- copier, distribuer et communiquer le matériel par tous moyens et sous tous formats.
    \end{itemize}

\bigskip

    Selon les conditions suivantes :
    \begin{itemize}
        \item \textbf{Attribution} : vous devez créditer l'\oe{}uvre, intégrer un lien vers la licence et indiquer si des modifications ont été effectuées à l'\oe{}uvre. Vous devez indiquer ces informations par tous les moyens raisonnables, sans toutefois suggérer que l'Offrant vous soutient ou soutient la façon dont vous avez utilisé son \oe{}uvre.
        \item \textbf{Pas d'utilisation commerciale} : vous n'êtes pas autorisé à faire un usage commercial de cette \oe{}uvre, tout ou partie du matériel la composant. 
        \item \textbf{Pas de modifications} : dans le cas où vous effectuez un remix, que vous transformez, ou créez à partir du matériel composant l'\oe{}uvre originale, vous n'êtes pas autorisé à distribuer ou mettre à disposition l'\oe{}uvre modifiée. 
    \end{itemize}

\medskip

L'Offrant ne peut retirer les autorisations concédées par la licence tant que vous appliquez les termes de cette licence.

\medskip

    Consulter la licence en entier pour plus de détails (\url{https://creativecommons.org/licenses/by-nc-nd/4.0/legalcode.fr}). 

\newpage
\thispagestyle{empty}
\tableofcontents

\nocite{*}
\newpage

\section{Monographies}

\printbibliography[heading=subbibliography,keyword=monographie2017,heading=none]

\section{Direction d'ouvrage et de numéro de revue}

\subsection{Direction d'ouvrage}

\printbibliography[heading=subbibliography,keyword=dirouvrage2017,heading=none]

\subsection{Direction de numéro de revue}

\printbibliography[heading=subbibliography,keyword=dirrevue2017,heading=none]

\section{Éditions}

\printbibliography[heading=subbibliography,keyword=editions2017,heading=none]

\section{Préfaces et postfaces}

\printbibliography[heading=subbibliography,keyword=prefpostface2017,heading=none]

\section{Articles}

\printbibliography[heading=subbibliography,keyword=article2017,heading=none]

\section{Chapitres d'ouvrage collectif}

\printbibliography[heading=subbibliography,keyword=chap2017,heading=none]

\section{Notices de dictionnaire ou d'encyclopédie}

\printbibliography[heading=subbibliography,keyword=artdictencyclo2017,heading=none]

\section{Recensions, compte rendus}

% \printbibliography[heading=subbibliography,keyword=compterendu2017,heading=none]

\section[Autres publications scientifiques]{Autres publications scientifiques : bases de données, rapports, blogs, catalogues d'exposition, traductions, etc.}

\printbibliography[heading=subbibliography,keyword=autre2017,heading=none]

\section{Travaux de vulgarisation}

\printbibliography[heading=subbibliography,keyword=vulgarisation2017,heading=none]

\end{document}
